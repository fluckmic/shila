\chapter{Related Work}
\label{chap:RelatedWork}

In this chapter, we provide the reader with points of reference for related work and technologies. %As far as we know there is no work directly comparable to this one.

A detailed review of the development of multipath transport protocols gives the survey \cite{MultiPathSurvey}. It discusses reasons for the paradigm shift from single to multipath and the challenges of this transition. Furthermore the Stream Control Transmission Protocol (SCTP) \cite{SCTPWiki, rfc4960} and Multipath TCP (MPTCP) \cite{Barre2011, Raiciu2012, MPTCPWebMain,rfc6824}, the two most famous protocols in these area, are investigated.

SCTP shares several characteristics with TCP \cite{rfc793}. Both protocols are reliable, connection-oriented and use a rate-based mechanism for Congestion Control. Unlike with TCP, with SCTP a connection can consist of multiple streams, either used in a multiplexed fashion or simultaneously. This prevents the protocol from suffering from the Head-of-line blocking problem \cite{HOLBlocking} which is an issue with TCP. SCTP furthermore supports multihoming, a feature originally only intended to be used for failover. However, works like Concurrent multipath transfer SCTP (CMT-SCTP) \cite{1709949} or wireless multipath transfer SCTP (WiMP-SCTP) \cite{8205908} propose efforts to benefit from the availability of multiple paths by using them concurrently. The Stream Control Transmission Protocol appeared when TCP had already established itself. Making a change to SCPT would require adjustments to the applications and network stack, the most important factors why the protocol was not successful \cite{WhyNotSCTP}.

Multipath TCP (MPTCP), as an extension to TCP, is the most recognized approach to multipathing. It can be used by all TCP applications without any changes to their implementation. With MPTCP, multiple paths are used for the data exchange within a single connection. This functionality for example provides advantages in the field of smartphone use. The protocol facilitates the combination of WLAN and cellular access \cite{MPTCPStudyWirelessAndCellular} or allows fluid handovers between individual connection points \cite{MPTCPStudyHandover}. Well-known smartphone vendors such as Apple, LG or Samsung rely on Multipath TCP in their devices for the pleasant switch between 4G and Wi-Fi \cite{MPTCPInSmartphones}. In addition to mobile phones, for example also data centers benefit from better performance thanks to MPTCP \cite{RBPGWH11}.

In the work \cite{MPQUICPaper}, the implementation of an extension for QUIC \cite{quic-transport-29, QUICChromium}, called Multipath QUIC (MPQUIC), is presented. QUIC itself is a relatively recently developed protocol running on top of UDP. Similar to SCTP, QUIC supports several streams but not the use of more than one path. MPQUIC enables the protocol to use multiple paths. An in-depth discussion of (MP)QUIC and an investigation of MPTCP with the focus on use cases with smartphones, devices predestined for multi-homing, is given in this thesis \cite{MPQUICThesis}.

Increased demands on reliability and performance as well as the availability of multiple network interfaces have led to the development of the mentioned multipath protocols.  Although the Internet was not originally designed for the use of multiple paths, it allows it.  This is contrary to new internet architectures such as NEBULA \cite{NEBULA} or SCION \cite{SCIONPaper, SCIONBook}, where multipathing is an inherent part of the design and a cornerstone of security and reliability. 

The TUN/TAB drivers \cite{TUNTAPDriver} provide accessibility of virtual network devices to applications in userland. This functionality is mainly used for tunneling, for example by VTun \cite{VTun} or OpenVPN \cite{OpenVPN}. 

Another approach to access network traffic in userland is  XDP \cite{XDP,XDPGitHub}, short for eXpress Data Path. It is not the first technology that allows the programmable processing of packets, but its approach differs from known solutions. For performance reasons, known products like the  DataPlane Development Kit (DPDK) \cite{DPDK} bypass the kernel with all its important security and isolation mechanisms at all. This stays in contrast to XDP where programmable packet processing is integrated directly into and verified by the kernel. A customizable XDP program is executed as part of the device driver on every packet. After parsing and modification within this so-called XDP driver hook, a packet can be dropped, pushed to the kernels network stack or redirected. One possible destination for redirection is AF\_XDP \cite{AFXDPPaper, AFXDPWeb,AFXDPWeb2}. AF\_XDP  is a special socket type that allows the exchange of raw packets between the network interface and a userland application at a high speed. 
